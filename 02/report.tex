\documentclass[a4paper]{article}

\usepackage[utf8]{inputenc}
\usepackage{amsmath}
\usepackage[french]{babel}
\usepackage{caption}
\usepackage{color}
\usepackage{enumitem}
\usepackage{etoolbox}
\usepackage{fancyhdr}
\usepackage[T1]{fontenc}
\usepackage[margin=2cm]{geometry}
\usepackage{graphicx}
\usepackage{hyperref}
\usepackage{listings}
\usepackage{tikz}

% Couleurs pour le code.
\definecolor{pgreen}{rgb}{0.0, 0.5, 0.0}
\definecolor{pred}{rgb}{0.9, 0.0, 0.0}

% Police utilisée pour le code.
\renewcommand{\ttdefault}{pcr}

% Espace avant et après une 'minipage'.
\BeforeBeginEnvironment{minipage}{\vskip 15pt}
\AfterEndEnvironment{minipage}{\vskip 10pt}

% Paramètres du paquet 'listings'.
\lstset{
	backgroundcolor = \color{white},
    basicstyle = \ttfamily,
    breakatwhitespace = false,
    breaklines = true,
    captionpos = none,
    columns = fixed,
    commentstyle = \color{pgreen},
    extendedchars = false,
    frame = trbl,
    frameround = none,
    framesep = 2pt,
    keywordstyle = \bfseries,
	language = java,
    numbers = left,
    numbersep = 5pt,
    numberstyle = \small\ttfamily,
    showspaces = false,
   	showstringspaces = false,
    stringstyle = \color{pred},
    tabsize = 4
}
% Fait en sorte que le code ne casse pas au milieu d'un saut de page.
\BeforeBeginEnvironment{lstlisting}{\begin{minipage}{\linewidth}}
\AfterEndEnvironment{lstlisting}{\end{minipage}}

% Désactive l'indentation par défaut des paragraphes.
\setlength\parindent{0pt}

% Suppression de la numérotation des sections.
%\setcounter{secnumdepth}{0}

% Profondeur de la table des matières.
\setcounter{tocdepth}{3}

\setitemize{itemsep=0.2em}

% Police par défaut.
\renewcommand{\familydefault}{\sfdefault}

\begin{document}

\makeatletter
\renewcommand{\@maketitle}{
	\newpage
    \null
    
    \begin{tikzpicture}[remember picture,overlay]
    	\node[anchor=north west,inner sep=1cm] at (current page.north west)
        	{\includegraphics[width=0.45\linewidth]{images/heigvd_logo.png}};
    \end{tikzpicture}
    
    \vfill
    
    \fontfamily{pag}
    
    \begin{center}
    	\@title
        \vskip 100pt
        {\Large \@author}
        \vskip 60pt
        {\large \@date}
    \end{center}
    
    \vfill
}
\makeatother

\title{
	{\fontsize{40pt}{40pt}\selectfont SYM}
    \vskip 1.5em
    {\Huge Laboratoire 02}
}

\author{
	\begin{tabular}{ll}
    	Étudiants: 	& Ludovic Delafontaine \\
    				& Lucas Elisei \\
        		   	& David Truan \\
        \\
        Professeur: & Fabien Dutoit \\
        Assistants:	& Michaël Sandoz \\
        			& Luca Bianchi
    \end{tabular}
}

% En-têtes.
\pagestyle{fancy}
\lhead{SYM}
\rhead{Laboratoire 01}

\maketitle

\pagebreak

\section{Introduction}

Le but de ce laboratoire est d'étudier la programmation répartie asynchrone à l'aide d'un serveur REST et d'implémenter de plusieurs manières la communication entre le client et le serveur à l'aide de simulation du monde réel.

Dans le premier cas, il s'agira de partir du principe que l'on reçoit une réponse quasi instantanée de la part du serveur.

Dans le second cas, on suppose qu'une coupure de réseau intervient durant l'utilisation de l'application (passage dans un tunnel, changement d'antenne réseau, etc.) et qu'il est nécessaire d'enregistrer et essayer à intervals réguliers de renvoyer les requêtes de l'utilisateur au serveur jusqu'à rétablissement et envoi avec succès des requêtes.

Dans le troisième cas 

\section{Réponses aux questions}

\begin{enumerate}
	\item Question 1
	
    \item Question 2
    
    \item Question 3
    
    \item Question 4
    
    \item Question 5
    
    \item Question 6
    
\end{enumerate}

\section{Conclusion}

Nous avons pu constater que l'implémentation des communications réseaux dans un contexte mobile n'est pas aussi simple que dans un contexte fixe. En effet, le changement fréquent d'environnements nous oblige à réfléchir à tous les cas et implémenter ceci de façon asynchrone afin de ne pas bloquer l'utilisation de l'application pendant la communication. De plus, à cause de ces contraintes de faible réseau, nous avons pu constater à nouveau qu'il était préférable d'envoyer des données les plus petites possibles et nous pouvons nous aider de la compression afin d'y parvenir.

\end{document}
