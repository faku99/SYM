\documentclass[a4paper]{article}

\usepackage[utf8]{inputenc}
\usepackage{amsmath}
\usepackage[french]{babel}
\usepackage{caption}
\usepackage{color}
\usepackage{enumitem}
\usepackage{etoolbox}
\usepackage{fancyhdr}
\usepackage[T1]{fontenc}
\usepackage[margin=2cm]{geometry}
\usepackage{graphicx}
\usepackage{hyperref}
\usepackage{listings}
\usepackage{tikz}

% Couleurs pour le code.
\definecolor{pgreen}{rgb}{0.0, 0.5, 0.0}
\definecolor{pred}{rgb}{0.9, 0.0, 0.0}

% Police utilisée pour le code.
\renewcommand{\ttdefault}{pcr}

% Espace avant et après une 'minipage'.
\BeforeBeginEnvironment{minipage}{\vskip 15pt}
\AfterEndEnvironment{minipage}{\vskip 10pt}

% Paramètres du paquet 'listings'.
\lstset{
	backgroundcolor = \color{white},
    basicstyle = \ttfamily,
    breakatwhitespace = false,
    breaklines = true,
    captionpos = none,
    columns = fixed,
    commentstyle = \color{pgreen},
    extendedchars = false,
    frame = trbl,
    frameround = none,
    framesep = 2pt,
    keywordstyle = \bfseries,
	language = java,
    numbers = left,
    numbersep = 5pt,
    numberstyle = \small\ttfamily,
    showspaces = false,
   	showstringspaces = false,
    stringstyle = \color{pred},
    tabsize = 4
}
% Fait en sorte que le code ne casse pas au milieu d'un saut de page.
\BeforeBeginEnvironment{lstlisting}{\begin{minipage}{\linewidth}}
\AfterEndEnvironment{lstlisting}{\end{minipage}}

% Désactive l'indentation par défaut des paragraphes.
\setlength\parindent{0pt}

% Suppression de la numérotation des sections.
%\setcounter{secnumdepth}{0}

% Profondeur de la table des matières.
\setcounter{tocdepth}{3}

\setitemize{itemsep=0.2em}

% Police par défaut.
\renewcommand{\familydefault}{\sfdefault}

\begin{document}

\makeatletter
\renewcommand{\@maketitle}{
	\newpage
    \null
    
    \begin{tikzpicture}[remember picture,overlay]
    	\node[anchor=north west,inner sep=1cm] at (current page.north west)
        	{\includegraphics[width=0.45\linewidth]{images/heigvd_logo.png}};
    \end{tikzpicture}
    
    \vfill
    
    \fontfamily{pag}
    
    \begin{center}
    	\@title
        \vskip 100pt
        {\Large \@author}
        \vskip 60pt
        {\large \@date}
    \end{center}
    
    \vfill
}
\makeatother

\title{
	{\fontsize{40pt}{40pt}\selectfont SYM}
    \vskip 1.5em
    {\Huge Laboratoire 03}
}

\author{
	\begin{tabular}{ll}
    	Étudiants: 	& Ludovic Delafontaine \\
    				& Lucas Elisei \\
        		   	& David Truan \\
        \\
        Professeur: & Fabien Dutoit \\
        Assistants:	& Michaël Sandoz \\
        			& Luca Bianchi
    \end{tabular}
}

% En-têtes.
\pagestyle{fancy}
\lhead{SYM}
\rhead{Laboratoire 03}

\maketitle

\pagebreak

\section{Introduction}

Le but de ce laboratoire est d'étudier l'utilisation des différents capteurs disponibles sur un mobile ainsi que des cas spécifiques d'utilisations.

Le premier cas d'utilisation est un système d'authentification à l'aide d'une combinaison d'un tag NFC et/ou d'un mot de passe.

Le seconde cas d'utilisation est le scan et l'interprétation de codes-barres en 1D et 2D.

Le troisième cas est l'utilisation de iBeacons qui peuvent être découverts dans l'environnement grâce au mobile.

Le dernier cas est l'utilisation des capteurs internes de base de la plateforme afin de réaliser une boussole.

\section{Réponses aux questions}

\begin{enumerate}
	\item Sachant que les collaborateurs de l'entreprise UBIQOMP SA se déplacent en véhiculant des
    informations précieuses dans leurs dispositifs informatiques mobiles (munis de dispositifs de lecture NFC), et qu'ils sont amenés à se rendre dans des zones à risque, un expert a fait les estimations suivantes : \\
    • La probabilité de vol d'un mobile par une personne malintentionnée et capable d'utiliser
    les données à des fins préjudiciables pour la société est de 1\% ; \\
    • La probabilité que le mot de passe puisse être découvert, soit par analyse des traces de
    doigts sur l'écran, soit par observation en cours d'utilisation est de 4\% ; \\
    • La probabilité de vol du porte-clés est de 0.1\% ; \\
    • Environ 10\% des criminels susceptibles d'accéder aux données du mobile sait que le porteclés
    permet l’accès au mobile. \\
    Quelle est la probabilité moyenne globale que des données soient perdues, dans le cas où il faut la
    balise ET le mot de passe, ainsi que dans le cas où il faut la balise OU le mot de passe (on négligera
    dans le calcul la probabilité de l’intersection des deux ensembles), ou encore le cas où seule la balise
    est nécessaire ? En d'autres termes, si l'on envoie cent collaborateurs en déplacement, quel est le
    risque encouru de vol de données sensibles ? Mettre vos conclusions en rapport avec l'inconfort
    subjectif de chaque solution.
    Peut-on améliorer la situation en introduisant un contrôle des informations d'authentification par un
    serveur éloigné (transmission d'un hash SHA256 du mot de passe et de la balise NFC) ? Si oui, à quelles
    conditions ? Quels inconvénients ?
    Proposer une stratégie permettant à la société UBIQOMP SA d'améliorer grandement son bilan
    sécuritaire, en détailler les inconvénients pour les utilisateurs et pour la société.
    
    \textbf{Réponse} \\
    REPONSE
	
    \item Comparer la technologie à codes-barres et la technologie NFC, du point de vue d'une utilisation dans des applications pour smartphones, dans une optique : \\
    • Professionnelle (Authentification, droits d’accès, clés de chiffrage) \\
    • Grand public (Billetterie, contrôle d’accès, e-paiement) \\
    • Ludique (Preuves d'achat, publicité, etc.) \\
    • Financier (Coûts pour le déploiement de la technologie, possibilités de recyclage, etc.)
    
    \textbf{Réponse} \\
	\begin{enumerate}

\item
Professionelle (Authentification, droits d’accès, clés de chiffrage):

Dans ce cadre, la technologie NFC est à préférer.

Certaines puces NFC permettent l'echahnge d'informations chiffrées, rendant l'utilisation du NFC préférable dans l'optique d'une authentification.

La technologie NFC est aussi plus ergonomique, ce qui est imporant dans un environement où l'authentificaton est fréquent.




\item
Grand public (Billetterie, contrôle d’accès, e-paiement):

Pour le grand public il est préférable d'utiliser les codes barres. En effet, pour le NFC, certains mobiles (Iphone) n'utilise pas cette technologie. C'est donc une grande partie des utilisateurs potentiels qui ne pourront pas profiter de notre application. Il est donc préférable d'utiliser les codes barres dont la technologie est plus répendue parmis les mobiles.

\item
Ludique (Preuves d'achat, publicité, etc.):

Il est plus facile de déployer des codes barres, par exemple sur des affiches pour de la publicité que des puces NFC.

Comme dit plus haut, comme tout les utilisateur ne dispose pas du NFC, une preuve d'achat se doit d'avoir les deux technologies, de ce fait, il est plus simple d'avoir uniquement la technologie la plus répendue.

\item
Financier (Coûts pour le déploiement de la technologie, possibilités de recyclage, etc.):

L'ajout d'un code barres à des supports est bon marché. Il nécessice peut-être un travail de conception supplémentaire mais l'impression à un coût minime.
A contrario, intégrer la technologie NFC peut être coûteuse étant donné les coûts de production et les efforts nécessaire pour integrer une puce NFC.


D'un point de vue recyclage, le NFC est re programmable pour une autre utilisation. Le code barre ne l'est pas!
   \end{enumerate}
    
    \item Les iBeacons sont très souvent présentés comme une alternative à NFC. Pouvez-vous commenter cette affirmation en vous basant sur 2-3 exemples de cas d'utilisations (use-cases) concrets (par exemple epaiement, second facteur d'identification, accéder aux horaires à un arrêt de bus, etc.).
    
    \textbf{Réponse} \\
    Les iBeacons présentent plusieurs avantages par rapport à la technologie NFC :
    \begin{itemize}
    		\item La portée : un iBeacon peut être détecté jusqu'à 50m, contre 10cm pour la technologie NFC.
    		\item Disponibilité : les iBeacons demandent la technologie BLE (Bluetooth Low Energy), disponible sur la grande majorité des smartphones récents, alors que NFC demande que le téléphone soit équipé d'une puce spéciale.
    \end{itemize}
    
    	Si on prend le cas d'utilisation d'un affichage des horaires à un arrêt de bus, la technologie iBeacon est plus intéressante car il suffit de se trouver à environ 50m pour pouvoir en profiter. De plus, imaginons que ce soit un arrêt très fréquenté, il serait dommage de devoir déranger la foule de personnes attendant leur bus dans le seul but d'accéder à la borne NFC. Cependant, dans le cas d'un paiement à la caisse d'un supermarché, la technologie NFC est préférable aux iBeacons car sa très courte portée apporte un sentiment de sécurité. L'administration du supermarché devra cependant investir d'avantage pour une borne NFC que pour un simple iBeacon.
    
    \item Une fois la manipulation effectuée, vous constaterez que les animations de la flèche ne sont pas
    fluides, il va y avoir un tremblement plus ou moins important même si le téléphone ne bouge pas.
    Veuillez expliquer quelle est la cause la plus probable de ce tremblement et donner une manière (sans
    forcément l’implémenter) d’y remédier
    
    \textbf{Réponse} \\
    Ces tremblements sont dû aux différentes imprécisions récupérées par les capteurs de champ magnétique et d'orientation. En effet, les capteurs sont parfois trop sensibles et enregistrent des données même lorsque l'appareil ne bouge pas (par exemple). Ces mesures parasites sont donc récupérées et interprétées "faussement" par l'application, ce qui induit les tremblements.
    
    Une façon d'atténuer - voir annuler - ces tremblements est d'utiliser des techniques de traitement du signal (un filtre passe bas par exemple) afin de réduire les valeurs parasites ou "lisser" les données perçues par les capteurs pour rendre l'animation fluide et sans tremblements.
    
\end{enumerate}

\section{Conclusion}

L'utilisation des différents capteurs du systme mobile n'est pas une chose aisée. En effet, l'interaction avec des périphériques hardware et l'environnement externe montre une précision parfois aléatoire dû au fait que l'environnement n'est pas toujours adapté à la situation (interférences électro-magnétiques, rebondissement des ondes, instabilité de communication, etc.).

Il sera nécessaire dans une application de production de prendre en compte ces différents cas et les gérer correctement afin de proposer une application de qualité pour l'utilisateur.

Néamoins, nous avons vu qu'à l'aide de librairies externes, le travail primaire de développement est grandement simplifié et qu'il n'est pas nécessaire de développer en bas niveau pour rapidement obtenir un résultat de la part des capteurs.

\end{document}
