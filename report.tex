\documentclass[a4paper]{article}

\usepackage{amsmath}
\usepackage[french]{babel}
\usepackage{caption}
\usepackage{color}
\usepackage{enumitem}
\usepackage{etoolbox}
\usepackage{fancyhdr}
\usepackage[T1]{fontenc}
\usepackage[margin=2cm]{geometry}
\usepackage{graphicx}
\usepackage{hyperref}
\usepackage{listings}
\usepackage{tikz}

% Couleurs pour le code.
\definecolor{pgreen}{rgb}{0.0, 0.5, 0.0}
\definecolor{pred}{rgb}{0.9, 0.0, 0.0}

% Police utilisée pour le code.
\renewcommand{\ttdefault}{pcr}

% Espace avant et après une 'minipage'.
\BeforeBeginEnvironment{minipage}{\vskip 15pt}
\AfterEndEnvironment{minipage}{\vskip 10pt}

% Paramètres du paquet 'listings'.
\lstset{
	backgroundcolor = \color{white},
    basicstyle = \ttfamily,
    breakatwhitespace = false,
    breaklines = true,
    captionpos = none,
    columns = fixed,
    commentstyle = \color{pgreen},
    extendedchars = false,
    frame = trbl,
    frameround = none,
    framesep = 2pt,
    keywordstyle = \bfseries,
	language = bash,
    numbers = left,
    numbersep = 5pt,
    numberstyle = \small\ttfamily,
    showspaces = false,
   	showstringspaces = false,
    stringstyle = \color{pred},
    tabsize = 4
}
% Fait en sorte que le code ne casse pas au milieu d'un saut de page.
\BeforeBeginEnvironment{lstlisting}{\begin{minipage}{\linewidth}}
\AfterEndEnvironment{lstlisting}{\end{minipage}}

% Désactive l'indentation par défaut des paragraphes.
\setlength\parindent{0pt}

% Suppression de la numérotation des sections.
%\setcounter{secnumdepth}{0}

% Profondeur de la table des matières.
\setcounter{tocdepth}{3}

\setitemize{itemsep=0.2em}

% Police par défaut.
\renewcommand{\familydefault}{\sfdefault}

\begin{document}

\makeatletter
\renewcommand{\@maketitle}{
	\newpage
    \null
    
    \begin{tikzpicture}[remember picture,overlay]
    	\node[anchor=north west,inner sep=1cm] at (current page.north west)
        	{\includegraphics[width=0.45\linewidth]{images/heigvd_logo.png}};
    \end{tikzpicture}
    
    \vfill
    
    \fontfamily{pag}
    
    \begin{center}
    	\@title
        \vskip 100pt
        {\Large \@author}
        \vskip 60pt
        {\large \@date}
    \end{center}
    
    \vfill
}
\makeatother

\title{
	{\fontsize{40pt}{40pt}\selectfont SYM}
    \vskip 1.5em
    {\Huge Laboratoire 01}
}

\author{
	\begin{tabular}{ll}
    	Étudiants: 	& Ludovic Delafontaine \\
    				& Lucas Elisei \\
        		   	& David Truan \\
        \\
        Professeur: & Fabien Dutoit \\
        Assistants:	& Michaël Sandoz \\
        			& Luca Bianchi
    \end{tabular}
}

% En-têtes.
\pagestyle{fancy}
\lhead{SYM}
\rhead{Laboratoire 01}

\maketitle

\pagebreak

\section{Introduction}

\section{Réponses}

\begin{enumerate}
	\item C'est dans les ressources. La langue par défaut prend la place de la traduction manquante.
	\item Clic-droit sur le dossier \lstinline{res}, \textbf{New > Image Asset}, choisir \textbf{- Icon} et Image pour Asset Type. Puis choisir l'image et l'éditer si besoin. Ceci va créer un dossier drawable.
\end{enumerate}

\end{document}